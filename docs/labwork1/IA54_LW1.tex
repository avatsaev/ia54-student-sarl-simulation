%----------------------------------------
% IT IS RECOMMENDED TO USE AUTOLATEX FOR
% COMPILING THIS DOCUMENT.
% http://www.arakhne.org/autolatex
%----------------------------------------

\documentclass[article,english,nodocumentinfo]{multiagentfrreport}

% The TeX code is entering with UTF8
% character encoding (Linux and MacOS standards)
\usepackage[utf8]{inputenc}
\usepackage{fancyhdr}
\usepackage{../common/sarl-colorized-listing}

\graphicspath{{imgs/auto/},{imgs/},{../common/}}

\declaredocument{Introduction to SARL}{Lab Works \#1}{UTBM-INFO-IA54-LW1}

\addauthorvalidator*[St\'ephane Galland]{St{\'e}phane}{Galland}{Teacher}

\updateversion{1.0}{\makedate{20}{01}{2016}}{First release on Github}{\upmpublic}

\Set[french]{institution}{Laboratoire Syst\`emes et Transport}
\Set[english]{institution}{Laboratoire Syst\`emes et Transport}

\Set{mafr_contact_name}{\phdname*{St\'ephane}{Galland}}
\Set{mafr_contact_email}{stephane.galland@utbm.fr}
\Set[french]{mafr_contact_phone}{03~84~58~34~18}
\Set[english]{mafr_contact_phone}{+33 384~583~418}

\newif\ifCODESKELETON
\CODESKELETONfalse

\newif\ifJANUSINCLASSPATH
\JANUSINCLASSPATHfalse

\begin{document}

\section{Goal of this Lab Work Session}

The goal of this lab work session is to introduce the basics concepts of agent-oriented programming with SARL.

You shall learn: 
\begin{itemize}
\item How to write the agent.
\item How to call an agent capacity.
\item How to spawn an agent from an agent.
\item How to define new capacities and skills.
\item How to exchange information between agents.
\end{itemize}

\section{First Step: prepare the development environment}

In this section, you will find the tasks to do for preparing your development environment.

\subsection{Installation of the Eclipse Tools}

\begin{emphbox}
Recommended version of SARL : 0.3.1-SNAPSHOT (stable) \\
Recommended version of Janus : 2.0.3.1-SNAPSHOT (release/stable version)
\end{emphbox}

\begin{enumerate}
\item Download the \Emph{Eclipse product} that contains the compilation tools for the SARL programming language : \url{http://www.sarl.io}
\item Uncompress the Eclipse product for SARL.
\item Download the \Emph{development version} of the Janus agent execution platform : \url{http://www.janusproject.io}
\item Do not uncompress the Jar file of the Janus platform.
\item Launch the downloaded Eclipse product.
\ifJANUSINCLASSPATH\else
\item You must ensure that the Eclipse environment contains a SARL run-time environment (SRE) by: \\
	\parbox{\linewidth}{\begin{enumerate}[a)]
	\item Open the Preferences dialog box, with the menu: \\
		\texttt{> Window > Preferences}
	\item Open the SARL section related to the SRE in the left side of the dialog box: \\
		\texttt{> SARL > Installed SREs}
	\item Check if a SRE is installed.
	\item If yes, close the Preferences dialog box.
	\item if no, add an SRE by clicking on "Add", and select the file of the Janus platform that you have download at the beginning of this section.
	\end{enumerate}}
\fi
\item Open the wizard for creating a SARL project, with the menu: \\
	\texttt{> File > New > Project > SARL > Project}
\item Enter the name of the project.
\ifJANUSINCLASSPATH
\item Click on the tab with the name \code{Libraries}.
\item In the list of the libraries, remove the \code{SARL Libraries}.
\item In the list of the libraries, add the download file of the Janus platform as an external Jar file.
\fi
\item Click on "Finish", the SARL project should be created.
\item You must ensure that the configuration of your SARL project is correct (may be still a bug in the SARL environment):
	\begin{enumerate}[a)]
	\item Open the dialog of the properties of the SARL project by clicking on: \\
		\texttt{Right click on project > Properties > SARL > Compiler > Output Folder}
	\item Check if the field ``Directory'' is set to a source folder that is existing in your SARL project. If not change the property with): \\
		\texttt{src/main/generated-sources/xtend}
	\end{enumerate}
\end{enumerate}

You Eclipse is now ready for the lab work session.
\ifCODESKELETON
You should now install the code skeleton provided by the teachers.

\subsection{Installation of the Code Skeleton}

The teachers provide a code skeleton that should be completed by you for terminating the tasks related to this lab work session.
The steps to follow for installing the code skeleton are:
\begin{enumerate}
\item Download from the Moodle of UTBM (\url{http://moodle.utbm.fr}) the file with the name \skeletonName.
\item Do not uncompress the downloaded Jar file.
\item Open the wizard for importing the code skeleton into the SARL project: \\
	\texttt{Right click on project > Import > General > Archive File}
\item Select on the local file system the downloaded file of the code skeleton; and click on ``Finish''.
\item The source folders of your project shall contains SARL and Java code. \\
	Some errors are appearing since they are related to the missed part of the code that must be provided by you. 
\item Clean the workspace for ensuring that each existing file is compiled: \\
	\texttt{Menu > Project > Clean}
\end{enumerate}

Figure \figref{eclipse_project_structure} on the page \figpageref{eclipse_project_structure} gives an example of the structure of the SARL project that you should obtain.

\mfigure[p]{width=.33\linewidth}{eclipse_project_structure}{Example of the structure of a SARL project}{eclipse_project_structure}
\fi

\subsection{SARL Documentation}

The documentation for the SARL syntax, and the provided elements is available at: \url{http://www.sarl.io/docs/suite/io/sarl/docs/SARLDocumentationSuite.html}

\subsection{Bug Report and Questions}

In case you have discovered an issue in the SARL tools, you could submit it to the SARL development team via the Github interface: \url{https://github.com/sarl/sarl}

In case you cannot discuss with your teacher, you could submit your questions to the SARL development team via the Github interface: \url{https://github.com/sarl/sarl}



\section{Work to be Done during the Lab Work Session}

The following sections describe the work to be done during this lab work session.

\subsection{First Agent}

First, you should write an agent that is able to display ``Hello World'' on the output console.

You should:
\begin{enumerate}[a)]
\item create the agent type with the name \code{Agent1} (with the wizard, or by hand).
\item create the event handler that is run when the \code{Initialize} event is received by the agent. This event is fired by the platform when the agent should initializee itself.
\item write the output statement in the event handler.
\end{enumerate}

For running the agent, you should:
\begin{enumerate}[a)]
\item Open the dialog box of the ``Run Configurations''.
\item On the left side, click on ``SARL Application''.
\item Click on the ``New launch configuration'' button.
\item On the right side, select the project and the agent to launch.
\item Click on ``Run''.
\end{enumerate}

\Emph{It is recommended to put a breakpoint in the event handler, and run the agent in debug mode.}

\subsection{Use the Logging Buildin Capacity}

The goal of this exercise is to use a capacity that is provided by the run-time platform, aka. buildin capacity.

You should:
\begin{enumerate}[a)]
\item update the agent code for using the ``Logging'' capacity.
\item May anything else be changed in the code?
\end{enumerate}

\subsection{Spawning Another Agent}

Most of the time, a system is composed by more than one agent. This exercise will enable you to launch another agent that is displaying ``Welcome'' on the output console also.

You should:
\begin{enumerate}[a)]
\item create a second agent \code{Agent2}, with its \code{Initialize} event handler that is displaying the welcome message.
\item Update the code of \code{Agent1} for using the \code{DefaultContextInteractions} buildin capacity. This capacity permits to do something with the context in which the agent is living, aka. the default context.
\item Update the \code{Initialize} event handler of \code{Agent1} for spawning an agent of type \code{Agent2}.
\item create a third agent \code{Agent3}, with its \code{Initialize} event handler that is displaying the welcome message.
\item Update the \code{Initialize} event handler of \code{Agent1} for spawning an agent of type \code{Agent3}.
\end{enumerate}

\subsection{Say Hello to Every Ones}

Agents are social entities. They are supposed to interact with other agents.
This exercice enables the \code{Agent1} to say hello to the other agents.

In SARL, the information exchanged between agents are carried out by events.
The events are put inside an interaction space of a context.
In this exercise, the default space of the default context will be used.
It is automatically accessible when using the \code{DefaultContextInteractions} capacity.

You should:
\begin{enumerate}[a)]
\item create the definition of the event \code{Hello}.
\item update the code of \code{Agent1} for sending the event to every one after it has spawned the other agents.
\item update the code of \code{Agent2} and \code{Agent3} for displaying the welcome message when they are receiving the hello event.
\end{enumerate}

\subsection{Say Hello to a Single Agent}

Broadcasting events may not be the interaction mode between two agents.
In this exercise, the \code{Hello} message will be sent only to \code{Agent3}.

You should:
\begin{enumerate}[a)]
\item update the code of \code{Agent1} for emiting the event with a scope retricted to \code{Agent3}.
\end{enumerate}

\subsection{Say Localized Welcome}

Agents may have different ways for doing a specific task.
In this exercise, the agents will be enable to say "Hello" according to their own skills, i.e. their languages.

The concepts of Capacity and Skill are suitable for this task.
A capacity is the definition of functions (similar to an interface in object-oriented programming) that could be invoked by the agents. A capacity never defines the code of a function, only its prototype.
A skill is a specific implementation of a capacity (similar to object implementing an interface in object-oriented programming). The skill must provide a code for each function of the implemented capacity.

You should:
\begin{enumerate}[a)]
\item define the \code{SayHello} capacity.
\item define the \code{SayHelloSkill} skill, that permits to say hello in English.
\item define the \code{DireBonjourSkill} skill, that permits to say hello in French.
\item update the code of the agents, that are defined previously, for using the \code{SayHello} capacity.
\item update the code of the agents for calling the function provided by the \code{SayHello} capacity.
\item update \code{Initialize} event handler of \code{Agent2} for defining its skill that is corresponding to the \code{SayHello} capacity. It should speak English.
\item update \code{Initialize} event handler of \code{Agent3} for defining its skill that is corresponding to the \code{SayHello} capacity. It should speak French.
\end{enumerate}

\subsection{Contract Net Protocol (optional)}

In this excercise, you must define a multiagent system that is running a Contract Net Protocol:
\begin{enumerate}
\item A customer wants to by a product.
\item The customer sends a request to a broker for finder the best provider.
\item The broker asks to a couple of provider they offers for the product.
\item The broker selects the best offer.
\item The broker notifies the selected provider about its acceptance with the ID of the customer.
\item The broker notifies the not selected providers about their rejections.
\item The broker notifies the customer with the ID of the provider.
\end{enumerate}

You should:
\begin{enumerate}[a)]
\item Define the capacity of the provider.
\item Define the capacities of the broker.
\item Write the provider code.
\item Write the broker code.
\item Write the customer code.
\end{enumerate}

\end{document}
