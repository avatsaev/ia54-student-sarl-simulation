%----------------------------------------
% IT IS RECOMMENDED TO USE AUTOLATEX FOR
% COMPILING THIS DOCUMENT.
% http://www.arakhne.org/autolatex
%----------------------------------------

\documentclass[article,english,nodocumentinfo]{multiagentfrreport}

% The TeX code is entering with UTF8
% character encoding (Linux and MacOS standards)
\usepackage[utf8]{inputenc}
\usepackage{fancyhdr}
\usepackage{../common/sarl-listing}

\graphicspath{{imgs/auto/},{imgs/},{../common/}}

\declaredocument{IA54/MABS Lab Work \#1}{Multiagent Simulation}{UTBM-INFO-IA54-MABS-LW1}

\addauthorvalidator*[St\'ephane Galland]{St{\'e}phane}{Galland}{Teacher}

\updateversion{1.0}{\makedate{13}{10}{2015}}{First release on Github}{\upmpublic}

\Set{mafr_contact_name}{\phdname*{St\'ephane}{Galland}}
\Set{mafr_contact_email}{stephane.galland@utbm.fr}
\Set[french]{mafr_contact_phone}{03~84~58~34~18}
\Set[english]{mafr_contact_phone}{+33 384~583~418}

\gdef\skeletonName{\texttt{\mbox{IA54\_MABS\_LW1\_skeleton\string.jar}}}

\begin{document}

\section{Goal of this Lab Work Session}

The goal of this lab work session is to write a simulator of the famous Pacman game.

You shall learn: 
\begin{itemize}
\item How to write the agent perception computation.
\item How to write the agent perception action mechanism.
\item How to write the Ghost behavior.
\item How to synchronize the agents.
\end{itemize}

\section{First Step: prepare the development environment}

In this section, you will find the tasks to do for preparing your development environment.

\subsection{Installation of the Eclipse Tools}

\begin{emphbox}
Recommended version of SARL : 0.3.1-SNAPSHOT (stable) \\
Recommended version of Janus : 2.0.3.1-SNAPSHOT (release/stable version)
\end{emphbox}

\begin{enumerate}
\item Download the \Emph{Eclipse product} that contains the compilation tools for the SARL programming language : \url{http://www.sarl.io}
\item Uncompress the Eclipse product for SARL.
\item Download the \Emph{development version} of the Janus agent execution platform : \url{http://www.janusproject.io}
\item Do not uncompress the Jar file of the Janus platform.
\item Launch the downloaded Eclipse product.
\ifJANUSINCLASSPATH\else
\item You must ensure that the Eclipse environment contains a SARL run-time environment (SRE) by: \\
	\parbox{\linewidth}{\begin{enumerate}[a)]
	\item Open the Preferences dialog box, with the menu: \\
		\texttt{> Window > Preferences}
	\item Open the SARL section related to the SRE in the left side of the dialog box: \\
		\texttt{> SARL > Installed SREs}
	\item Check if a SRE is installed.
	\item If yes, close the Preferences dialog box.
	\item if no, add an SRE by clicking on "Add", and select the file of the Janus platform that you have download at the beginning of this section.
	\end{enumerate}}
\fi
\item Open the wizard for creating a SARL project, with the menu: \\
	\texttt{> File > New > Project > SARL > Project}
\item Enter the name of the project.
\ifJANUSINCLASSPATH
\item Click on the tab with the name \code{Libraries}.
\item In the list of the libraries, remove the \code{SARL Libraries}.
\item In the list of the libraries, add the download file of the Janus platform as an external Jar file.
\fi
\item Click on "Finish", the SARL project should be created.
\item You must ensure that the configuration of your SARL project is correct (may be still a bug in the SARL environment):
	\begin{enumerate}[a)]
	\item Open the dialog of the properties of the SARL project by clicking on: \\
		\texttt{Right click on project > Properties > SARL > Compiler > Output Folder}
	\item Check if the field ``Directory'' is set to a source folder that is existing in your SARL project. If not change the property with): \\
		\texttt{src/main/generated-sources/xtend}
	\end{enumerate}
\end{enumerate}

You Eclipse is now ready for the lab work session.
\ifCODESKELETON
You should now install the code skeleton provided by the teachers.

\subsection{Installation of the Code Skeleton}

The teachers provide a code skeleton that should be completed by you for terminating the tasks related to this lab work session.
The steps to follow for installing the code skeleton are:
\begin{enumerate}
\item Download from the Moodle of UTBM (\url{http://moodle.utbm.fr}) the file with the name \skeletonName.
\item Do not uncompress the downloaded Jar file.
\item Open the wizard for importing the code skeleton into the SARL project: \\
	\texttt{Right click on project > Import > General > Archive File}
\item Select on the local file system the downloaded file of the code skeleton; and click on ``Finish''.
\item The source folders of your project shall contains SARL and Java code. \\
	Some errors are appearing since they are related to the missed part of the code that must be provided by you. 
\item Clean the workspace for ensuring that each existing file is compiled: \\
	\texttt{Menu > Project > Clean}
\end{enumerate}

Figure \figref{eclipse_project_structure} on the page \figpageref{eclipse_project_structure} gives an example of the structure of the SARL project that you should obtain.

\mfigure[p]{width=.33\linewidth}{eclipse_project_structure}{Example of the structure of a SARL project}{eclipse_project_structure}
\fi

\subsection{SARL Documentation}

The documentation for the SARL syntax, and the provided elements is available at: \url{http://www.sarl.io/docs/suite/io/sarl/docs/SARLDocumentationSuite.html}

\subsection{Bug Report and Questions}

In case you have discovered an issue in the SARL tools, you could submit it to the SARL development team via the Github interface: \url{https://github.com/sarl/sarl}

In case you cannot discuss with your teacher, you could submit your questions to the SARL development team via the Github interface: \url{https://github.com/sarl/sarl}



\section{Brief Description of the Code Skeleton}

The skeleton contains an implementation of the environment model (a maze) \texttt{fr.utbm.info.ia54.environment}, and the agent that is managing the environment.
The subpackages are or will be:
\begin{itemize}
\item \texttt{fr.utbm.info.ia54.environment.maze} is the package that contains the definition of the maze, and the objects inside.
\item \texttt{fr.utbm.info.ia54.environment.agent} is the package in which you could find the definition of the agent and the interaction space that are needed for supported the maze environment in the simulation.
\end{itemize}

\emph{It is recommended to read this code and the associated Javadoc.}

The package \texttt{fr.utbm.info.ia54.players} contains the playing agents, i.e. the ghosts.

The packages \texttt{fr.utbm.info.ia54.math} and \texttt{fr.utbm.info.ia54.ui} contain mathematical tools and the graphical user interface.
The file \texttt{fr/utbm/info/ia54/PacManSimulator.java} contains the main program.

\section{Work to be Done during the Lab Work Session}

The following sections describe the work to be done during this lab work session.

\subsection{Capacity to Manage a Maze}

The agent \texttt{Environment} is in charge of managing the maze and the interactions with the playing agents (the ghosts), and the pacman.
The agent \texttt{Environment} has the knowledge of the maze. 
It means that this agent has the capacity to manage the maze.
This capacity is \texttt{MazeManager}, defined in the file \texttt{fr/utbm/info/ia54/environment/agent/Capacities.sarl}.


You must create a skill, i.e. an implementation of a capacity, for the capacity \texttt{MazeManager}.
In the code skeleton, this skill is named \texttt{DefaultMazeManagerSkill}, and all the functions inside must be code.
The skill contains a reference to the maze.

You should:
\begin{enumerate}[a)]
\item implement the functions in \texttt{DefaultMazeManagerSkill} that implements the functions defined in \texttt{MazeManager}.
\end{enumerate}

Two functions may be overriden for coding the skill:
\begin{itemize}
\item \texttt{def install}: it is invoked when the agent is starting to use the skill; and
\item \texttt{def uninstall}: it is invoked when the agent is stopping to use the skill.
\end{itemize}

\subsection{Capacity to be the Front-End to Players}

The agent \texttt{Environment} is also in charge of providing the perceptions to the playing agents, i.e. the ghosts, and to accept the actions from them.
It means that this agent has the capacity to be the front-end to the player.
This capacity is \texttt{MazeFrontEnd}, defined in the file \texttt{fr/utbm/info/ia54/environment/agent/Capacities.sarl}.

You should:
\begin{enumerate}[a)]
\item implement the functions in \texttt{DefaultMazeFrontEndSkill} that implements the functions defined in \texttt{MazeFrontEnd}.
\end{enumerate}

Be sure that your capacity is creating an interaction space from the space specification \texttt{MazeSpaceSpecification}, and with the proper ID.

You could take a look in the definition of \texttt{DefaultMazeMotionSkill} that is the counter part (in the players) of the \texttt{DefaultMazeFrontEndSkill} capacity.

\subsection{Ghost Behavior}

The agent \texttt{Ghost} must be defined in the file \texttt{fr/utbm/info/ia54/players/Players.sarl}

This agent should:
\begin{itemize}
\item Pursue the pacman if it is in the field-of-view of the ghost, and the pacman has no super power.
\item Evade the pacman if it is in the field-of-view of the ghost, and the pacman has the super power.
\item Move randomly, but avoiding to turn back in the middle of a corridor.
\end{itemize}

\end{document}
